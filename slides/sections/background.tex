\section{Background}

\begin{frame}{Disclaimer!}
    When I say vertex, point, or vector, I (most likely) am referring to the same thing.
\end{frame}

\subsection{Motivation}

\begin{frame}{Why bother with ANNS Graphs?}
    \begin{itemize}
        \item High-dimensional vector data sets are becoming more popular (especially embedding real-world information into vectors)
        \item One use case for these data sets is for similarity search, with real world applications such as
            \begin{itemize}
                \item Recommendation systems
                \item Information retrieval systems
            \end{itemize}
        \item Tree-based and Hash-based methods are only useful with low-dimensional vector data due to \textit{the curse of dimensionality}
        \item Graph-based data structures are the best performing so far
    \end{itemize}
\end{frame}

\begin{frame}{Why bother with vertex removal?}
    \begin{itemize}
        \item Data can become outdated so there is no point to storing stale data
        \item Existing data structures only support marking vertices as deleted 
        \item Existing data structures also usually implements an upper limit to how many points can be stored in a graph
    \end{itemize}
\end{frame}

\begin{frame}
    \textbf{Goal:} Design a data structure that supports effecient vertex removal.
\end{frame}

\subsection{Definitions}

\begin{frame}{\(k\)-Nearest Neighbor Search}
\begin{definition}[\(k\)-Nearest Neighbor Search]
    Given the point set \(\mathcal{P} \subset \mathbb{R}^d\), query point \(q \in \mathbb{R}^d\), and a distance function \(\delta\), we want to find a set \(\mathcal{K} \subseteq \mathcal{P}\) such that \(|\mathcal{K}| = k\) and
    \[
        \max_{p \in \mathcal{K}} \delta(p, q) \leq \min_{p \in \mathcal{P} \setminus \mathcal{K}} \delta(p, q)
    \]
    \end{definition}

    Finding exact \(\mathcal{K}\) is expensive. So, we generally find the approximate \(\mathcal{K}\) instead.
\end{frame}

\begin{frame}{Approximation Quality}
\begin{definition}[Recall]
    Given the point set \(\mathcal{P} \subset \mathbb{R}^d\) and a query point \(q \in \mathbb{R}^d\), let \(\mathcal{K} \subset \mathcal{P}\) be \(q\)'s true \(k\) nearest neighbors in \(\mathcal{P}\) and \(\mathcal{N}\) be the output of an ANNS algorithm with \(|\mathcal{N}| = n\). We define \textit{\(k\)-recall at \(n\) of \(q\)} as
    \[
        \frac{|\mathcal{K} \cap \mathcal{N}|}{|\mathcal{K}|}
    \]
    This is denoted as \(k@n\) recall.
\end{definition}
\end{frame}

\subsection{Existing Solutions}

\begin{frame}{Existing Solutions}
    There are a few approaches to solving \(k\)-ANNS:
    \begin{itemize}
        \item Naive linear search (very expensive)
        \item Hash-based like LSH (does not work in high \(d\))
        \item Tree-based like \(k\)D-trees (also does not work in high \(d\))
        \item Graph-based (where the good stuff is)
    \end{itemize}
\end{frame}
