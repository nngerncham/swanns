\section{Experiments}

\begin{frame}{Approximation Quality}
\begin{definition}[Recall]
    Given the point set \(\mathcal{P} \subset \mathbb{R}^d\) and a query point \(q \in \mathbb{R}^d\), let \(\mathcal{K} \subset \mathcal{P}\) be \(q\)'s true \(k\) nearest neighbors in \(\mathcal{P}\) and \(\mathcal{N}\) be the output of an ANNS algorithm with \(|\mathcal{N}| = n\). We define \textit{\(k\)-recall at \(n\) of \(q\)} as
    \[
        \frac{|\mathcal{K} \cap \mathcal{N}|}{|\mathcal{K}|}
    \]
    This is denoted as \(k@n\) recall.
\end{definition}
\end{frame}

\begin{frame}{Which should we use as base?}
    Want a data structure with efficient vertex removal along with good construction time and search? Build on top of existing ones!
    \begin{itemize}
        \item Both HNSW and DiskANN are very promising 
        \item DiskANN seems to be better according to one paper
        \item HNSW is still more popular than DiskANN
    \end{itemize}
\end{frame}

\subsection{Experiment Setup}

\begin{frame}{Implementations}
    The implementations of the algorithms used are
    \begin{itemize}
        \item \texttt{hnswlib} (HNSW), implemented by the authors of HNSW
        \item \texttt{faiss} (HNSW), implemented by Facebook AI
        \item \texttt{diskannpy} (DiskANN), implemented by the authors of DiskANN
    \end{itemize}
\end{frame}

\begin{frame}{Data Sets}
    The data sets used for the experiments are
    \begin{itemize}
        \item SIFT1M, a million 128-dimension vectors
        \item GIST1M, SIFT1M but 960-dimension
    \end{itemize}
\end{frame}

\subsection{Results}

\begin{frame}{SIFT1M Results}
    
\end{frame}

\begin{frame}{GIFT1M Results}
    
\end{frame}

\subsection{Conclusion}

\begin{frame}{Which is better?}
    Basically, \textit{it depends} on what your trade-offs are.
    \begin{itemize}
        \item HNSW takes up more memory
        \item DiskANN requires significantly more build time to get high accuracy
    \end{itemize}
\end{frame}
